\vspace*{\fill}
\thispagestyle{empty}

{\large \centering \textbf{Sommario} \par}

\vspace{1em} % Some space between title and abstract

% Justified text for the abstract
\noindent
Nell'ultimo decennio, è stato possibile assistere ad un'importante crescita nella diffusione delle applicazioni basate su microservizi. Questo perché strutturare un'applicazione a microservizi comporta svariati vantaggi, ma anche una maggiore complessità dell'architettura. Data questa maggiore complessità, l'abilità di analizzare e raccogliere dati inerenti a potenziali fallimenti risulta particolarmente importante. L'obiettivo di questa tesi è quindi quello di semplificare l'analisi dei fallimenti, fornendo un supporto per tenere automaticamente traccia delle interazioni tra i microservizi di un'applicazione e il loro esito. In particolare, si fornisce una soluzione per l'estensione di un dispiegamento Kubernetes, con l'obiettivo di raccogliere i log delle interazioni tra microservizi in maniera \enquote{black-box}, ovvero non conoscendo/modificando il codice delle applicazioni associate. La tesi mostra anche un'applicazione pratica della soluzione a diverse applicazioni esistenti.
\vspace*{\fill}